\section{Anhang}
\label{sec:anhang}

\subsection{Quellcode der Baseline-Modelle}
\label{app:source_code_baselines}

Der folgende Quellcode zeigt die PyTorch-Implementierung der verwendeten Baseline-Modelle \textit{ConcatBERT} und \textit{SumBERT}. Beide Klassen erben von \texttt{BertForSequenceClassification} und passen den Forward-Pass entsprechend der jeweiligen Fusionsstrategie an.

\begin{lstlisting}[language=Python, caption={PyTorch-Implementierung von ConcatBERT und SumBERT}, label={lst:baseline_code}, basicstyle=\ttfamily\footnotesize, breaklines=true, keywordstyle=\color{blue}, stringstyle=\color{green!50!black}, commentstyle=\color{gray}]
class ConcatBert(BertForSequenceClassification):
    """Custom BERT that handles numerical and one-hot encoded categorical data by concatenating it to the output embedding
    of the CLS token.

    Attributes:
    ----------
    classifier: ClassificationHead
        A classification module that is identical to the one in BertForSequenceClassification, except for added input dimensions from the numerical data.

    Methods:
    ---------
    forward(...):
        Takes in input ids, attention mask, and a tensor of extra data to compute the binary output. Input ids and attention mask are first fed through the base bert model.
        The resulting CLS context embedding is then concatenated to the extra data tensor and fed through the classificationHead to get the final output.
    """

    def __init__(self, cfg: BertConfig, extra_data_dim: int, hidden_dim: int):
        super().__init__(cfg)

        class ClassificationHead(nn.Module):
            """Head for sentence-level classification tasks."""

            def __init__(self, cfg, extra_data_dim, hidden_dim):
                super().__init__()
                total_dims = cfg.hidden_size + extra_data_dim
                assert (
                    hidden_dim <= total_dims
                ), "The hidden dim should be less or equal than the input embedding"
                self.dense = nn.Linear(total_dims, hidden_dim)
                self.bn1 = nn.BatchNorm1d(hidden_dim)
                classifier_dropout = (
                    cfg.classifier_dropout
                    if cfg.classifier_dropout is not None
                    else cfg.hidden_dropout_prob
                )
                self.dropout = nn.Dropout(classifier_dropout)
                self.out_proj = nn.Linear(hidden_dim, cfg.num_labels)

            def forward(self, features, **kwargs):
                x = self.dropout(features)
                x = self.dense(x)
                x = self.bn1(x)
                x = torch.relu(x)
                x = self.dropout(x)
                x = self.out_proj(x)
                return x

        self.classifier = ClassificationHead(cfg, extra_data_dim, hidden_dim)
        # https://tfs/web/DefaultCollection/GIT_Projects/_git/hf4-regress-exploration/pullRequest/81216#1737385342
        self.post_init()

    def forward(
        self,
        input_ids: torch.Tensor | None = None,
        attention_mask: torch.Tensor | None = None,
        token_type_ids: torch.Tensor | None = None,
        position_ids: torch.Tensor | None = None,
        head_mask: torch.Tensor | None = None,
        inputs_embeds: torch.Tensor | None = None,
        labels: torch.Tensor | None = None,
        output_attentions: bool | None = None,
        output_hidden_states: bool | None = None,
        return_dict: bool | None = None,
        extra_data: torch.Tensor | None = None,
    ) -> tuple | SequenceClassifierOutput:

        return_dict = (
            return_dict if return_dict is not None else self.config.use_return_dict
        )

        outputs = self.bert(
            input_ids,
            attention_mask=attention_mask,
            token_type_ids=token_type_ids,
            position_ids=position_ids,
            head_mask=head_mask,
            inputs_embeds=inputs_embeds,
            output_attentions=output_attentions,
            output_hidden_states=output_hidden_states,
            return_dict=return_dict,
        )
        # Using the pooler output as document embedding. This corresponds
        # to a post-processed version of the <CLS> token embedding.
        cls_embedding = outputs.pooler_output

        if extra_data is not None:
            output = torch.cat((cls_embedding, extra_data), dim=-1)
        else:
            raise ValueError("Must supply extra data!")
        logits = self.classifier(output)
        if not return_dict:
            output = (logits,) + outputs[2:]
            return output

        return SequenceClassifierOutput(
            logits=logits,
            hidden_states=outputs.hidden_states,
            attentions=outputs.attentions,
        )


class SumBert(BertForSequenceClassification):
    """
    SumBert integrates additional feature-embeddings with a BERT base model used for a classification task.

    Parameters
    ----------
    feature_layer:
      Linear layer used to add the numerical and categorical data to the word embedding of the CLS token.

    Methods
    ---------
    forward(...):
        Gets word embeddings from input ids through bert. The adds the output of its feature layer to the CLS word embedding to insert numerical / categorical data.
        Subsequently uses BertModels forward function to get outputs.
    """

    def __init__(self, cfg: BertConfig, extra_data_dim: int):
        super().__init__(cfg)
        self.feature_layer = nn.Linear(extra_data_dim, 768)

    def forward(
        self,
        input_ids: torch.Tensor | None = None,
        attention_mask: torch.Tensor | None = None,
        token_type_ids: torch.Tensor | None = None,
        position_ids: torch.Tensor | None = None,
        head_mask: torch.Tensor | None = None,
        inputs_embeds: torch.Tensor | None = None,
        labels: torch.Tensor | None = None,
        output_attentions: bool | None = None,
        output_hidden_states: bool | None = None,
        return_dict: bool | None = None,
        extra_data: torch.Tensor | None = None,
    ):
        # Avoid performing computation under torch.no_grad() to ensure
        # word embeddings can be updated during training.
        text_embedding = self.bert.embeddings.word_embeddings(input_ids)
        text_embedding[:, 0] += self.feature_layer(extra_data)
        outputs = self.bert(
            inputs_embeds=text_embedding,
            attention_mask=attention_mask,
        )
        pooled_output = outputs.pooler_output
        pooled_output = self.dropout(pooled_output)
        logits = self.classifier(pooled_output)
        return SequenceClassifierOutput(
            logits=logits,
            hidden_states=outputs.hidden_states,
            attentions=outputs.attentions,
        )
\end{lstlisting}
